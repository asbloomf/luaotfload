\input luaotfload.sty

%%%%%%%%%%%%%%%%%%%%%%%%%%%%%%%%%%%%%%%%%%%%%%%%%%%%%%%%%%%%%%%%%%%%%%%
%% unicode character mappings
%%%%%%%%%%%%%%%%%%%%%%%%%%%%%%%%%%%%%%%%%%%%%%%%%%%%%%%%%%%%%%%%%%%%%%%

\font\ptserifregular = file:PTF55F.ttf \ptserifregular

%% here we map the function luaotfload.aux.name_of_slot
%% on a short text, printing a list of letters, their
%% code points and names (as specified in the Adobe
%% Glyph List).

\directlua{
  local aux = luaotfload.aux
  local cbk = function (str)
    if string.match(str, "^EOF") then
      luatexbase.remove_from_callback("process_input_buffer", "weird")
      return [[the end!]]
    end
    local res = { }
    for chr in string.utfcharacters(str) do
      local val = unicode.utf8.byte(chr)
      local line = chr .. " <> " .. tostring(val)
      line = line .. " <> " .. (aux.name_of_slot(val) or "")
      res[\string#res+1] = line
    end
    return table.concat(res, [[\endgraf]])
  end

  luatexbase.add_to_callback("process_input_buffer", cbk, "weird")
}

Я узнал что у меня
Есть огромная семья
И тропинка и лесок
В поле каждый колосок

EOF

\bye
