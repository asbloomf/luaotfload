% \iffalse meta-comment
%
% Copyright (C) 2009-2013 by Elie Roux    <elie.roux@telecom-bretagne.eu>
%                        and Khaled Hosny <khaledhosny@eglug.org>
%                             (Support: <lualatex-dev@tug.org>.)
%
% This work is under the CC0 license.
%
% This work consists of the main source file luaotfload.dtx
% and the derived files
%    luaotfload.sty, luaotfload.lua
%
% Unpacking:
%    tex luaotfload.dtx
%
% Documentation:
%    lualatex luaotfload.dtx
%
%    The class ltxdoc loads the configuration file ltxdoc.cfg
%    if available. Here you can specify further options, e.g.
%    use A4 as paper format:
%       \PassOptionsToClass{a4paper}{article}
%
%
%
%<*ignore>
\begingroup
  \def\x{LaTeX2e}%
\expandafter\endgroup
\ifcase 0\ifx\install y1\fi\expandafter
         \ifx\csname processbatchFile\endcsname\relax\else1\fi
         \ifx\fmtname\x\else 1\fi\relax
\else\csname fi\endcsname
%</ignore>
%<*install>
\input docstrip.tex
\Msg{************************************************************************}
\Msg{* Installation}
\Msg{* Package: luaotfload v2.2 OpenType layout system}
\Msg{************************************************************************}

\keepsilent
\askforoverwritefalse

\let\MetaPrefix\relax

\preamble
This is a generated file.

Copyright (C) 2009-2013 by by Elie Roux    <elie.roux@telecom-bretagne.eu>
                          and Khaled Hosny <khaledhosny@eglug.org>
                               (Support: <lualatex-dev@tug.org>.)

This work is under the CC0 license.

This work consists of the main source file luaotfload.dtx
and the derived files
    luaotfload.sty, luaotfload.lua

\endpreamble

\let\MetaPrefix\DoubleperCent

\generate{%
  \usedir{tex/luatex/luaotfload}%
  \file{luaotfload.sty}{\from{luaotfload.dtx}{package}}%
}

% The following hacks are to generate a lua file with lua comments starting with
% -- instead of %%

\def\MetaPrefix{-- }

\def\luapostamble{%
  \MetaPrefix^^J%
  \MetaPrefix\space End of File `\outFileName'.%
}

\def\currentpostamble{\luapostamble}%

\generate{%
  \usedir{tex/luatex/luaotfload}%
  \file{luaotfload.lua}{\from{luaotfload.dtx}{lua}}%%
}

\obeyspaces
\Msg{************************************************************************}
\Msg{*}
\Msg{* To finish the installation you have to move the following}
\Msg{* files into a directory searched by TeX:}
\Msg{*}
\Msg{*     luaotfload.sty, luaotfload.lua}
\Msg{*}
\Msg{* Happy TeXing!}
\Msg{*}
\Msg{************************************************************************}

\endbatchfile
%</install>
%<*ignore>
\fi
%</ignore>
%<*driver>
\NeedsTeXFormat{LaTeX2e}
\ProvidesFile{luaotfload.drv}%
  [2013/04/16 v2.2 OpenType layout system]%
\documentclass{ltxdoc}
\usepackage{metalogo,multicol,mdwlist,fancyvrb,xspace}
\usepackage[x11names]{xcolor}
%
\def\primarycolor{DodgerBlue4}  %%-> rgb  16  78 139 | #104e8b
\def\secondarycolor{Goldenrod4} %%-> rgb 139 105 200 | #8b6914
%
\usepackage[
    bookmarks=true,
   colorlinks=true,
    linkcolor=\primarycolor,
     urlcolor=\secondarycolor,
    citecolor=\primarycolor,
     pdftitle={The luaotfload package},
   pdfsubject={OpenType layout system for Plain TeX and LaTeX},
    pdfauthor={Elie Roux & Khaled Hosny},
  pdfkeywords={luatex, lualatex, unicode, opentype}
]{hyperref}
\usepackage{fontspec}
%usepackage{unicode-math}%% broken
\setmainfont[Numbers=OldStyle,Ligatures=TeX]{Linux Libertine O}
\setmonofont[Ligatures=TeX,Scale=MatchLowercase]{Liberation Mono}
%setsansfont[Ligatures=TeX]{Linux Biolinum O}
\setsansfont[Ligatures=TeX,Scale=MatchLowercase]{Iwona Medium}
%setmathfont{XITS Math}

\newcommand\TEX    {\TeX\xspace}
\newcommand\LUA    {Lua\xspace}
\newcommand\PDFTEX {pdf\TeX\xspace}
\newcommand\LUATEX {Lua\TeX\xspace}
\newcommand\XETEX  {\XeTeX\xspace}
\newcommand\LATEX  {\LaTeX\xspace}
\newcommand\CONTEXT{Con\TeX t\xspace}

\def\definehighlight[#1][#2]%
 {\ifcsname #1\endcsname\else
    \expandafter\def\csname #1\endcsname%
      {\bgroup#2\csname #1_indeed\endcsname}
    \expandafter\def\csname #1_indeed\endcsname##1%
      {##1\egroup}%
  \fi}

\def\restoreunderscore{\catcode`\_=12\relax}

\definehighlight    [fileent][\ttfamily\restoreunderscore]         %% files, dirs
\definehighlight   [texmacro][\sffamily\itshape\textbackslash]     %% cs
\definehighlight[luafunction][\sffamily\itshape\restoreunderscore] %% lua identifiers
\definehighlight [identifier][\sffamily]                           %% names
\definehighlight     [abbrev][\rmfamily\scshape]                   %% acronyms
\definehighlight   [emphasis][\rmfamily\slshape]                   %% level 1 emph

\newcommand*\email[1]{\href{mailto:#1}{#1}}

\VerbatimFootnotes
\begin{document}
  \DocInput{luaotfload.dtx}%
\end{document}
%</driver>
% \fi
%
% \CheckSum{0}
%
% \CharacterTable
%  {Upper-case    \A\B\C\D\E\F\G\H\I\J\K\L\M\N\O\P\Q\R\S\T\U\V\W\X\Y\Z
%   Lower-case    \a\b\c\d\e\f\g\h\i\j\k\l\m\n\o\p\q\r\s\t\u\v\w\x\y\z
%   Digits        \0\1\2\3\4\5\6\7\8\9
%   Exclamation   \!     Double quote  \"     Hash (number) \#
%   Dollar        \$     Percent       \%     Ampersand     \&
%   Acute accent  \'     Left paren    \(     Right paren   \)
%   Asterisk      \*     Plus          \+     Comma         \,
%   Minus         \-     Point         \.     Solidus       \/
%   Colon         \:     Semicolon     \;     Less than     \<
%   Equals        \=     Greater than  \>     Question mark \?
%   Commercial at \@     Left bracket  \[     Backslash     \\
%   Right bracket \]     Circumflex    \^     Underscore    \_
%   Grave accent  \`     Left brace    \{     Vertical bar  \|
%   Right brace   \}     Tilde         \~}
%
% \GetFileInfo{luaotfload.drv}
%
% \title{The \identifier{luaotfload} package}
% \date{2013/04/16 v2.2}
% \author{Elie Roux and Khaled Hosny\\
%         Support: \email{lualatex-dev@tug.org}}
%
% \maketitle
%
% \begin{abstract}
% This package is an adaptation of the \CONTEXT font loading system, providing
% the ability to load \identifier{OpenType} fonts with extended font loading syntax
% supporting a large selection of \identifier{OpenType} font features.
% \end{abstract}
%
% \tableofcontents
%
% \section{Introduction}
%
% Font management and installation has always been painful with \TEX.  A lot of
% files are needed for one font (\abbrev{tfm}, \abbrev{pfb}, \abbrev{map},
% \abbrev{fd}, \abbrev{vf}), and as \TEX is 8-bit each font is limited to 256
% characters.
% But the font world has evolved since the original
% \TEX, and new typographic systems have appeared, most notably the so
% called \emphasis{smart font} technologies like \identifier{OpenType}
% (\abbrev{otf}) fonts.
% These fonts can contain a lot of characters and additional
% functionality like ligatures, old-style numbers, small capitals,
% etc., and support more complex writing systems like Arabic and
% Indic\footnote{%
%   Unfortunately, \identifier{luaotfload} doesn't support Indic
%   scripts right now.
%   Assistance in implementing the prerequisites is greatly
%   appreciated.
% }
% scripts.
% \identifier{OpenType} fonts are widely deployed and available for all
% modern operating systems.
% As of 2013 they have become the de facto standard for advanced text
% layout.
% However, until recently the only way to use them directly in the \TEX
% world was with the \XETEX engine.
%
% Unlike \XETEX, \LUATEX has no built-in support for
% \identifier{OpenType} or other technologies.
% Instead, it provides hooks for executing Lua during the \TEX run
% that allow implementing extensions for loading fonts and manipulating
% how input text is processed without modifying the underlying engine.
%
% \section{Loading fonts}
%
% \identifier{luaotfload} supports an extended font loading syntax:
%
% \begin{center}
%       |\font\foo={|%
%       \meta{prefix}|:|%
%       \meta{font name}|:|%
%       \meta{font features}|}|%
%       \meta{\TEX font features}
% \end{center}
%
% \noindent
% The curly brackets are optional and escape the spaces in the enclosed
% font name (alternatively, double quotes serve the same purpose).
% The individual parts of the syntax are:
%
% \paragraph{Prefix}
%
% The \meta{prefix} is either |file:| or |name:|.
% It determines whether font loader should interpret the request as a
% file name or font name, respectively, which again influences how it
% will attempt to locate the font.
% The prefix can be omitted, in which case |name:| is assumed.
%
%% \iffalse%% how am i supposed to friggin comment stuff in a dtx???
%%      TODO
%%      it would appear that the next paragraph is incorrect; I get
%%      name: lookups regardless unless the font file is actually
%%      in CWD
%% \fi
%% For compatibility with \XETEX, surrounding the \meta{font name} with
%% square brackets is synonymous to using the |file:| prefix.
%
% Accessing fonts by fontname allows loading system installed fonts as
% well as \fileent{texmf} ones, and requires a font names database; see
% Section~\ref{sec:fontdb} for more information.
%
% \paragraph{Font name}
%
% The \meta{font name} can be either a font filename or actual font
% name based on the \meta{prefix} as mentioned above.
%
% Fonts loaded by filename may either include their absolute path in
% the filesystem or consist of just the filename without a path.  If no
% path is specified, then \identifier{kpathsea} is used to locate the
% font (which will typically be in the \fileent{texmf} tree or the
% current directory).
%
% For example,
% \begin{quote}
% \begin{verbatim}
% \font\1={file:ec-lmr10} at 10pt
% \font\2={/Users/Shared/Fonts/aldus.otf} at 11pt
% \font\3={name:TeX Gyre Pagella} at 9pt
% \end{verbatim}
% \end{quote}
%
%
% \paragraph{Font features}
%
% \meta{font features} is semicolon-separated list of feature
% tags\footnote{%
%   Cf. \url{http://www.microsoft.com/typography/otspec/featurelist.htm}.
% }
% and font options.
% Prepending a font feature with a |+|-sign enables it, while
% a |-| disables it. For instance, the request
%
% |\font\test=Latin Modern Roman:+clig;-kern|
%
% \noindent activates contextual ligatures (|clig|) and disables
% kerning (|kern|).
% Alternatively the options |true| or |false| can be passed to
% the feature in a key/value expression.
% The following request has the same meaning as the last one:
%
% |\font\test=Latin Modern Roman:clig=true;kern=false|
%
% \noindent
% Furthermore, this second syntax is required if a font feature
% accepts options besides its activation state.
% For example, \emphasis{stylistic alternates} (|salt|) provide a set
% of variants to given glyphs.
% These can be selected either explicitly by supplying the variant
% index (starting from 1), or randomly by setting the value to,
% obviously, |random|:
%
% |\font\test=Latin Modern Roman:salt=1|
%
% \noindent Other font options include:
%
% \begin{description}
%
% \item [mode] \hfill \\
%        \identifier{luaotfload} has two \identifier{OpenType} processing
%        \emphasis{modes}:
%        \identifier{base} and \identifier{node}.
%
%        \identifier{base} mode works by mapping \identifier{OpenType}
%        features to traditional \TEX ligature and kerning mechanisms,
%        thus supporting only non-contextual substitutions and kerning
%        pairs, but is the slightly faster variant.
%        \identifier{node} mode works by processing \TEX’s internal
%        node list directly at the \LUA end and supports
%        a wider range of \identifier{OpenType} features.
%        The downside is that the intricate operations required for
%        \identifier{node} mode may slow down typesetting especially
%        with complex fonts and it does not work in math mode.
%
%        By default \identifier{luaotfload} is in \identifier{node}
%        mode, and \identifier{base} mode has to be requested where needed,
%        e.~g. for math fonts.
%
% \item [script] \ref{script-tag} \hfill \\
%        An \identifier{OpenType} script tag;\footnote{%
%          See \url{http://www.microsoft.com/typography/otspec/scripttags.htm}
%          for a list of valid values.
%          For scripts derived from the Latin alphabet the value
%          |latn| is good choice.
%        }
%        the default value is |dlft|.
%        Some fonts do not assign features to the |dflt| script, in
%        which case the script needs to be set explicitly.
%
% \item [language] \hfill \\
%        An \identifier{OpenType} language system identifier,\footnote{%
%          Cf. \url{http://www.microsoft.com/typography/otspec/languagetags.htm}.
%        }
%        defaulting to |dflt|.
%
% \item [featurefile] \hfill \\
%        A comma-separated list of feature files to be applied to the
%        font.
%        Feature files contain a textual representation of
%        \identifier{OpenType} tables and extend the features of a font
%        on fly.
%        After they are applied to a font, features defined in a
%        feature file can be enabled or disabled just like any
%        other font feature.
%        The syntax is documented in \identifier{Adobe}’s
%        \identifier{OpenType} Feature File Specification.\footnote{%
%          Cf. \url{http://www.adobe.com/devnet/opentype/afdko/topic_feature_file_syntax.html}.
%        }
%
%        For a demonstration of how to set a |tkrn| feature consult
%        the file |tkrn.fea| that is part of \identifier{luaotfload}.
%        It can be read and applied as follows:
%
%        |\font\test=Latin Modern Roman:featurefile=tkrn.fea;+tkrn|
%
% \item [color] \hfill \\
%        A font color, defined as a triplet of two-digit hexadecimal
%        \abbrev{rgb} values, with an optional fourth value for
%        transparency
%        (where |00| is completely transparent and |FF| is opaque).
%
%        For example, in order to set text in semitransparent red:
%
%        |\font\test=Latin Modern Roman:color=FF0000BB|
%
% \item [protrusion \& expansion] \hfill \\
%        These keys both control microtypographic features of the font,
%        namely \emphasis{character protrusion} and \emphasis{font
%        expansion}.
%        They accept names of predefined \LUA tables that contain
%        values for the respective features.\footnote{%
%           For examples of the table layout please refer to the
%           section of the file \fileent{otfl-fonts-ext.lua} where the
%           default values are defined.
%           Alternatively and with loss of information, you can dump
%           those tables into your terminal by issuing
%           \begin{verbatim}
%             \directlua{inspect(fonts.protrusions.setups.default)
%                        inspect(fonts.expansions.setups.default)}
%           \end{verbatim}
%           at some point after loading \fileent{luaotfload.sty}.
%        }
%        For both, only the set \identifier{default} is predefined.
%
%        For example, to enable default protrusion\footnote{%
%          You also need to set
%              \verb|pdfprotrudechars=2| and
%              \verb|pdfadjustspacing=2|
%          to activate protrusion and expansion, respectively.
%          See the
%          \href{http://mirrors.ctan.org/systems/pdftex/manual/pdftex-a.pdf}%
%               {\PDFTEX manual}
%          for details.
%        }:
%
%        |\font\test=Latin Modern Roman:protrusion=default|
% \end{description}
%
% \paragraph{Non-standard font features}
% \identifier{luaotfload} add a number of features that are not defined
% in the original \identifier{OpenType} specification, most of them
% aiming at emulating the behavior familiar from other \TEX engines.
% Currently (2013) there are three of them:
%
% \begin{description}
%
%   \item [anum]
%           Substitutes the glyphs in the \abbrev{ascii} number range
%           with their counterparts from eastern Arabic or Persian,
%           depending on the value of \identifier{language}.
%
%   \item [tlig]
%           Applies legacy \TEX ligatures:
%           |``|, |''|, |`|, |'|, |"|, |--|, |---|, |!`| and |?`|.%
%           \footnote{%
%             These contain the feature set \verb|trep| of earlier
%             versions of \identifier{luaotfload}.
%
%             Note to \XETEX users: this is the equivalent of the
%             assignment \verb|mapping=text-tex| using \XETEX's input
%             remapping feature.
%           }
%
%   \item [itlc]
%           Computes italic correction values (active by default).
%
% \end{description}
%
%
%
% \section{Font names database}
% \label{sec:fontdb}
%
% As mentioned above, \identifier{luaotfload} keeps track of which
% fonts are available to \LUATEX by means of a \emphasis{database}.
% This allows loading not only by explicit filenames but also by the
% proper names contained in the metadata which is often more accessible
% to humans.\footnote{%
%   The tool \href{http://www.lcdf.org/type/}{\fileent{otfinfo}} (comes
%   with \TEX Live), when invoked on a font file with the \verb|-o|
%   option, lists the variety of name fields defined for it.
% }
%
% When \identifier{luaotfload} is asked to load a font by font name, it
% will check if the database exists and load it or else generate a
% fresh one.
% Should it then fail to locate the font, an update to the database is
% performed in case the font has been added to the system only
% recently.  As soon as the database is updated, the resolver will try
% and look up the font again, all without user intervention.
% The goal is for \identifier{luaotfload} to act in the background and
% behave as unobtrusively as possible, while providing a convenient
% interface to the fonts installed on the system.
%
% \subsection[mkluatexfontdb.lua]%
%            {\fileent{mkluatexfontdb.lua}\footnote{%
%   The script may be named just \fileent{mkluatexfontdb} in your
%   distribution.
% }}
%
% However, it can be desirable at times to do some of these steps
% manually.
% To this end, \identifier{luaotfload} comes with the utility
% \fileent{mkluatexfontdb} that offers an interface to the database
% functionality.
% Being a \LUA script, there are two ways to run it:
% either make it executable (\verb|chmod +x| on unixoid systems) or
% pass it as an argument to \fileent{texlua}.\footnote{%
%   Tests by the maintainer show only marginal performance gain by
%   running with Luigi Scarso’s
%   \href{https://foundry.supelec.fr/projects/luajittex/}%
%        {\LUA jit\TEX},
%   which is probably due to the fact that most of the time is spent
%   on file system operations.
%
%   \emphasis{Note}:
%   On \abbrev{MS} \identifier{Windows} systems, the script can be run
%   either by calling the wrapper application
%   \fileent{mkluatexfontdb.exe} or as
%   \verb|texlua.exe mkluatexfontdb.lua|.
% }
% Invoke it from the command line with the \verb|--force| switch to
% initiate a complete rebuild of the database.
%
% \begin{verbatim}
%   mkluatexfontdb --force
% \end{verbatim}
%
% Generating the database for the first time may take a while since it
% inspects every font file on your computer.
% This is particularly noticeable if it occurs during a typesetting run.
% In any case, subsequent updates to the database will be quite fast.
%
% \subsection{Search Paths}
%
% \identifier{luaotfload} scans those directories where fonts are
% expected to be located on a given system.
% On a Linux machine it follows the paths listed in the
% \identifier{Fontconfig} configuration files;
% consult \verb|man 5 fonts.conf| for further information.
% On \identifier{Windows} systems, the standard location is
% \verb|Windows\Fonts|,
% while \identifier{Mac OS~X} requires a multitude of paths to
% be examined.
% The complete list is is given in table \ref{table-searchpaths}.
% Other paths can be specified by setting the environment variable
% \verb+OSFONTDIR+.
% If it is non-empty, then search will be limited to the included
% directories.
%
% \begin{table}[t]
%   \hrule
%   \caption{List of paths searched for each supported operating
%            system.}
%   \renewcommand{\arraystretch}{1.2}
%   \begin{center}
%     \begin{tabular}{lp{.5\textwidth}}
%       Windows     & \verb|%WINDIR%\Fonts|
%       \\
%       Linux       & \fileent{/usr/local/etc/fonts/fonts.conf} and\hfill\break
%                     \fileent{/etc/fonts/fonts.conf"}
%       \\
%       Mac         & \fileent{~/Library/Fonts},\break
%                     \fileent{/Library/Fonts},\break
%                     \fileent{/System/Library/Fonts}, and\hfill\break
%                     \fileent{/Network/Library/Fonts}
%       \\
%     \end{tabular}
%   \end{center}
%   \label{table-searchpaths}
%   \hrule
% \end{table}
%
% \subsection{Querying from outside}
%
% \fileent{mkluatexfontdb.lua} also provides rudimentary means of
% accessing the font database.
% If the option \verb|--find=name| is given, the script will try and search
% the fonts indexed by \identifier{luaotfload} for a matching name.
% For instance, the invocation
%
% \begin{verbatim}
%   mkluatexfontdb.lua  --find="Iwona Regular"
% \end{verbatim}
%
% will verify if “Iwona Regular” is found in the database and can be
% readily requested in a document.
%
% If you are unsure about the actual font name, then you can add the
% \verb|-F| switch to the command line to enable approximate matching.
% Suppose you cannot precisely remember if the variant of
% \identifier{Iwona} you are looking for was “Bright” or “Light”.
% The query 
%
% \begin{verbatim}
%   mkluatexfontdb.lua  -F --find="Iwona Bright"
% \end{verbatim}
%
% will tell you that indeed the latter name is correct.
%
% \verb|mkluatexfontdb.lua --help| will list the available command line
% switches, including some that will not be discussed in detail here.
%
% \subsection{Blacklisting fonts}
% \label{font-blacklist}
%
% Some fonts are problematic in general, or just in \LUATEX.
% If you find that compiling your document takes far too long or eats
% away all your system’s memory, you can track down the culprit by
% running \verb|mkluatexfontdb -v| to increase verbosity.
% Take a note of the \emphasis{filename} of the font that database
% creation fails with and append it to the file
% \fileent{otfl-blacklist.cnf}.
%
% A blacklist file is a list of font filenames, one per line.
% Specifying the full path where the file is located is optional, the
% plain filename should suffice.
% File extensions (\fileent{.otf}, \fileent{.ttf}, etc.) may be omitted.
% Anything after a percent (|%|) character until the end of the line
% is ignored, so use this to add comments.
% Place this file to some location where the \identifier{kpse}
% library can find it, e.~g.
% \fileent{texmf-local/tex/luatex/luaotfload} if you are running
% \identifier{\TEX Live},\footnote{%
%   You may have to run \verb|mktexlsr| if you created a new file in
%   your \fileent{texmf} tree.
% }
% or just leave it in the working directory of your document.
% \identifier{luaotfload} reads all files named
% \fileent{otfl-blacklist.cnf} it finds, so the fonts in
% \fileent{./otfl-blacklist.cnf} extend the global blacklist.
%
% Furthermore, a filename prepended with a dash character (|-|) is
% removed from the blacklist, causing it to be temporarily whitelisted
% without modifying the global file.
% An example with explicit paths:
%
% \begin{verbatim}
% % example otf-blacklist.cnf
% /Library/Fonts/GillSans.ttc  % Luaotfload ignores this font.
% -/Library/Fonts/Optima.ttc   % This one is usable again, even if
%                              % blacklisted somewhere else.
% \end{verbatim}
%
% \section{Files from \CONTEXT and \LUATEX-Fonts}
%
% This package relies on code originally written by Hans
% Hagen\footnote{%
%   The creator of the \href{http://wiki.contextgarden.net}{\CONTEXT}
%   format.
% }
% for and tested with \CONTEXT.
% \identifier{luaotfload} integrates the font loader as distributed in
% the \identifier{\LUATEX-Fonts} package.
% The original \LUA source files have been combined using the
% \fileent{mtx-package} script into a single, self-contained blob.
% In this form the font loader has no further dependencies\footnote{%
%   It covers, however, to some extent the functionality of the
%   \identifier{lualibs} package.
% }
% and requires only minor adaptions to integrate into
% \identifier{luaotfload}.
% The guiding principle is to let \CONTEXT/\LUATEX-Fonts take care of
% the implementation, and update the imported code from time to time.
% As maintainers, we aim at importing files from upstream essentially
% \emphasis{unmodified}, except for renaming the files to prevent name
% clashes.
% This job has been greatly alleviated since the advent of
% \LUATEX-Fonts, prior to which the individual dependencies had to be
% manually spotted and extracted from the \CONTEXT source code in a
% complicated and error-prone fashion.
%
% Below is a commented list of the files distributed with
% \identifier{luaotfload} in one way or the other.
% See figure \ref{file-graph} on page \pageref{file-graph} for a
% graphical representation of the dependencies.
% From \LUATEX-Fonts the following files have been imported:
%
%   \begin{itemize}
%     \let\normalitem=\item
%     \def\fileitem#1#2{%
%       \normalitem{\fileent{#1}}%
%       \hfill
%       (as \fileent{\itshape#2})%
%       \break
%     }
%     \def\incitem#1{%
%       \normalitem{\fileent{#1}}
%     }
%     \fileitem{luatex-fonts.lua}{otfl-fonts.lua}
%           The wrapper that loads the font loader code.
%
%     \fileitem{luatex-fonts-merged.lua}{otfl-fonts-merged.lua}
%           The font loader package.
%           It is generated by \fileent{mtx-package}, a \LUA
%           source code merging tool developed by Hans
%           Hagen.\footnote{%
%             \fileent{mtx-package} is
%             \href
%               {http://repo.or.cz/w/context.git/blob_plain/refs/heads/origin:/scripts/context/lua/mtx-package.lua}
%               {part of \CONTEXT}
%             and requires \fileent{mtxrun}.
%             Run
%             \verb|mtxrun --script package --help|
%             to display further information.
%             For the actual merging code see the file
%             \fileent{util-mrg.lua} that is part of \CONTEXT.
%           }
%
%           Included are several Lua files that can be classed in three
%           categories.
%           \begin{itemize}
%             \normalitem \emphasis{\LUA utility libraries}, a subset
%                         of what is provided by the \identifier{lualibs}
%                         package.
%
%                         \begin{multicols}{2}
%                           \begin{itemize}
%                             \incitem{l-lua.lua}       \incitem{l-lpeg.lua}
%                             \incitem{l-function.lua}  \incitem{l-string.lua}
%                             \incitem{l-table.lua}     \incitem{l-io.lua}
%                             \incitem{l-file.lua}      \incitem{l-boolean.lua}
%                             \incitem{l-math.lua}      \incitem{util-str.lua}
%                           \end{itemize}
%                         \end{multicols}
%
%             \normalitem The \emphasis{Font Loader} itself.
%                         These files have been written for
%                         \LUATEX-Fonts and are distributed along with
%                         \identifier{luaotfload}.
%                         \begin{multicols}{2}
%                           \begin{itemize}
%                             \incitem{luatex-basics-gen.lua}
%                             \incitem{luatex-basics-nod.lua}
%                             \incitem{luatex-fonts-enc.lua}
%                             \incitem{luatex-fonts-syn.lua}
%                             \incitem{luatex-fonts-tfm.lua}
%                             \incitem{luatex-fonts-chr.lua}
%                             \incitem{luatex-fonts-lua.lua}
%                             \incitem{luatex-fonts-def.lua}
%                             \incitem{luatex-fonts-ext.lua}
%                             \incitem{luatex-fonts-cbk.lua}
%                           \end{itemize}
%                         \end{multicols}
%
%             \normalitem Code related to \emphasis{font handling and
%                         node processing}, taken directly from
%                         \CONTEXT.
%                         \begin{multicols}{2}
%                           \begin{itemize}
%                             \incitem{data-con.lua} \incitem{font-ini.lua}
%                             \incitem{font-con.lua} \incitem{font-cid.lua}
%                             \incitem{font-map.lua} \incitem{font-oti.lua}
%                             \incitem{font-otf.lua} \incitem{font-otb.lua}
%                             \incitem{node-inj.lua} \incitem{font-ota.lua}
%                             \incitem{font-otn.lua} \incitem{font-def.lua}
%                           \end{itemize}
%                         \end{multicols}
%           \end{itemize}
%
%   \end{itemize}
%
% In addition to these, \identifier{luaotfload} requires a number of
% files not contained in the merge. Some of these have no equivalent in
% \LUATEX-Fonts or \CONTEXT, some were taken unmodified from the
% latter.
%
% \begin{itemize}
%     \let\normalitem=\item
%     \def\ouritem#1{%
%       \normalitem{\fileent{#1}}%
%       \space--\hskip1em
%     }
%     \ouritem {otfl-font-otc.lua} \fileent{font-otc} from \CONTEXT;
%                                  font feature handling.
%     \ouritem {otfl-lib-dir.lua}  \fileent{l-dir} from \CONTEXT;
%                                  contains functionality required
%                                  by \fileent{otfl-font-nms.lua}.
%     \ouritem {otfl-luat-ovr.lua} overrides for the \CONTEXT logging
%                                  functionality.
%     \ouritem {otfl-font-pfb.lua} registers the \identifier{OpenType}
%                                  font reader as handler for
%                                  Postscript fonts.
%     \ouritem {otfl-font-nms.lua} font database.
%     \ouritem {otfl-font-clr.lua} color handling.
%     \ouritem {otfl-font-ltx.lua} font feature handling.
%     \ouritem {otfl-features.lua} definitions of the \verb|anum| and
%                                  \verb|tlig| features.
% \end{itemize}
%
% \begin{figure}[b]
%   \caption{Schematic of the Files in \identifier{Luaotfload}}
%   \includegraphics[width=\textheight,angle=90]{filegraph.pdf}
%   \label{file-graph}
% \end{figure}
%
% \section{Troubleshooting}
%
% If you encounter problems with some fonts, please first update to the latest
% version of this package before reporting a bug, as
% \identifier{luaotfload} is under active development and still a
% moving target.
% Errors during database generation can be traced by increasing
% verbosity levels and redirecting log output to \fileent{stdout}:
%
% \begin{verbatim}
%   mkluatexfontdb.lua -F -vvv --log=stdout
% \end{verbatim}
%
% If this fails, the font last printed to the terminal is likely to be
% the culprit.
% Please specify it when reporting a bug, and blacklist it for the time
% being (see above, page \pageref{font-blacklist}).
%
% A common problem is the lack of features for some
% \identifier{OpenType} fonts even when specified.
% This can be related to the fact that some fonts do not provide
% features for the |dflt| script (see above on page
% \pageref{script-tag}),
% which is the default one in this package.
% If this happens, assigning a script when the font is defined should
% fix it.
% For example with the |latn| script:
%
% \begin{verbatim}
%   \font\test=file:MyFont.otf:script=latn;+liga;
% \end{verbatim}
%
% \part{\fileent{luaotfload.lua}}
%
% \iffalse
%<*lua>
% \fi
%    \begin{macrocode}
luaotfload = luaotfload or {}

luaotfload.module = {
    name          = "luaotfload",
    version       = 2.2,
    date          = "2013/04/15",
    description   = "OpenType layout system.",
    author        = "Elie Roux & Hans Hagen",
    copyright     = "Elie Roux",
    license       = "CC0"
}

local luatexbase = luatexbase

local type, next       = type, next
local stringfind       = string.find
local stringsub        = string.sub
local stringmatch      = string.match
local find_file        = kpse.find_file

local add_to_callback, create_callback =
      luatexbase.add_to_callback, luatexbase.create_callback
local reset_callback, call_callback =
      luatexbase.reset_callback, luatexbase.call_callback

local dummy_function = function () end

%    \end{macrocode}
%
%    No final decision has been made on how to handle font definition.  At
%    the moment, there are three candidates: The \identifier{generic}
%    callback as hard-coded in the font loader, the \identifier{old}
%    wrapper, and a simplified version of the latter (\identifier{patch})
%    that does nothing besides applying font patches.
%
%    \begin{macrocode}
luaotfload.font_definer = "patch" --- | “generic” | “old”

local error, warning, info, log =
    luatexbase.provides_module(luaotfload.module)

%    \end{macrocode}
%
%    This is a necessary initalization in order not to rebuild an existing
%    font.
%    Maybe 600 should be replaced by \texmacro{pdfpkresolution} %% (why?)
%    or \luafunction{texconfig.pk_dpi} (and it should be replaced
%    dynamically), but we don't have access (yet) to the
%    \identifier{texconfig} table, so we let it be 600.
%    Anyway, it does still work fine even if \texmacro{pdfpkresolution} is
%    changed.
%
%    \begin{macrocode}

kpse.init_prog("", 600, "/")

%    \end{macrocode}
%
%    We set the minimum version requirement for \LUATEX to v0.74, as it was
%    the first version to include version 5.2 of the \LUA interpreter.
%
%    \begin{macrocode}

local luatex_version = 74

if tex.luatexversion < luatex_version then
    warning("LuaTeX v%.2f is old, v%.2f is recommended.",
             tex.luatexversion/100,
             luatex_version   /100)
end

%    \end{macrocode}
%
%    \subsection{Module loading}
%
%    We load the files imported from \CONTEXT with this function.
%    It automatically prepends the prefix \fileent{otfl-} to its argument,
%    so we can refer to the files with their actual \CONTEXT name.
%
%    \begin{macrocode}

local fl_prefix = "otfl" -- “luatex” for luatex-plain
local loadmodule = function (name)
    require(fl_prefix .."-"..name)
end

%    \end{macrocode}
%
%    Before \TeX Live 2013 version, \LUATEX had a bug that made ofm fonts fail
%    when called with their extension. There was a side-effect making ofm
%    totally unloadable when luaotfload was present. The following lines are
%    a patch for this bug. The utility of these lines is questionable as they
%    are not necessary since \TeX Live 2013. They should be removed in the next
%    version.
%
%    \begin{macrocode}
local Cs, P, lpegmatch = lpeg.Cs, lpeg.P, lpeg.match

local p_dot, p_slash = P".",  P"/"
local p_suffix       = (p_dot * (1 - p_dot - p_slash)^1 * P(-1)) / ""
local p_removesuffix = Cs((p_suffix + 1)^1)

local find_vf_file = function (name)
    local fullname = find_file(name, "ovf")
    if not fullname then
        --fullname = find_file(file.removesuffix(name), "ovf")
        fullname = find_file(lpegmatch(p_removesuffix, name), "ovf")
    end
    if fullname then
        log("loading virtual font file %s.", fullname)
    end
    return fullname
end

--[[-- keep --]]
--- from Hans (all merged):

---   file name              modified  include name
--- × basics-gen.lua         t         luat-basics-gen
--- × font-def -> fonts-def  t         luatex-font-def (there’s also the normal font-def!)
--- × fonts-enc              f         luatex-font-enc
--- × fonts-ext              t         luatex-fonts-ext
--- × fonts-lua              f         luatex-fonts-lua
---   fonts-tfm              f         luatex-fonts-tfm
--- × fonts-cbk              f         luatex-fonts-lua

--- from Hans (unmerged):
---   font-otc.lua -> otfl-font-otc.lua

--- from luaotfload:
---   otfl-luat-ovr.lua    -- override some luat-dum functions
---   otfl-font-clr.lua
---   otfl-font-ltx.lua
---   otfl-font-nms.lua
---   otfl-font-pfb.lua    -- ?

--[[-- new --]]
--- basics-nod          (merged as fonts-nod !)
--- fonts-demo-vf-1.lua
--- fonts-syn           (merged)

--[[-- merged, to be dropped --]]
--- otfl-data-con.lua
--- otfl-font-cid.lua
--- otfl-font-con.lua
--- otfl-font-ini.lua
--- otfl-font-ota.lua
--- otfl-font-otb.lua
--- otfl-font-otf.lua
--- otfl-font-oti.lua
--- otfl-font-otn.lua

%    \end{macrocode}
%
%
%    \subsection{Preparing the Font Loader}
%    We treat the fontloader as a black box so behavior is consistent
%    between formats.
%    The wrapper file is \fileent{otfl-fonts.lua} which we imported from
%    \href{http://standalone.contextgarden.net/current/context/experimental/tex/generic/context/luatex/}{\LUATEX-Plain}.
%    It has roughly two purposes:
%
%    \begin{enumerate}
%
%       \item insert the functionality required for fontloader; and
%
%       \item put it in place via the respective callbacks.
%
%    \end{enumerate}
%
%    How the first step is executed depends on the presence on the
%    \emphasis{merged font loader code}.
%    In \identifier{luaotfload} this is contained in the file
%    \fileent{otfl-fonts-merged.lua}.
%    If this file cannot be found,  the original libraries from \CONTEXT of
%    which the merged code was composed are loaded instead.
%
%    Hans provides two global tables to control the font loader:
%
%      \begin{itemize}
%        \item  \luafunction{generic_context}:
%               encapsulation mechanism, callback functions
%        \item  \luafunction{non generic_context}:
%               customized code insertion
%      \end{itemize}
%
%    With \luafunction{non_generic_context} we can tailor the font loader
%    insertion to our file naming habits (key \luafunction{load_before}).
%    Additionally, \luafunction{skip_loading} can be unset to force loading
%    of the original libraries as though the merged code was absent.
%    Another key, \luafunction{load_after} is called at the time when the
%    font loader is actually inserted.
%    In combination with the option \luafunction{no_callbacks_yet} in
%    \luafunction{generic_context}, we can insert our own,
%    \identifier{luatexbase}-style callback handling here.
%
%    \begin{macrocode}
generic_context = {no_callbacks_yet = true}

non_generic_context = { luatex_fonts = {
        load_before     = "otfl-fonts-merged.lua",
        -- load_after      = nil, --- TODO, this is meant for callbacks
        skip_loading    = true,
}}

%    \end{macrocode}
%
%    The imported font loader will call \luafunction{callback.register} once
%    while reading \fileent{font-def.lua}.
%    This is unavoidable unless we modify the imported files, but harmless
%    if we make it call a dummy instead.
%
%    \begin{macrocode}

local trapped_register = callback.register
callback.register      = dummy_function

%    \end{macrocode}
%
%    In order to have an output with normal verbosity, we need to pre-override
%    some \CONTEXT log function:
%
%    \begin{macrocode}

local dummylogfunction=function() end
local dummylogreporter=function(c) return function(...) log(string.formatters(...)) end end

logs={
  new=dummylogreporter,
  reporter=dummylogreporter,
  messenger=dummylogreporter,
  report=dummylogfunction,
}

%    \end{macrocode}
%
%    Now that things are sorted out we can finally load the fontloader.
%
%    \begin{macrocode}

loadmodule"fonts.lua"

%    \end{macrocode}
%
%    By default, the fontloader requires a number of \emphasis{private
%    attributes} for internal use.
%    These must be kept consistent with the attribute handling methods as
%    provided by \identifier{luatexbase}.
%    Our strategy is to override the function that allocates new attributes
%    before we initialize the font loader, making it a wrapper around
%    \luafunction{luatexbase.new_attribute}.\footnote{%
%        Many thanks, again, to Hans Hagen for making this part
%        configurable!
%    }
%    The attribute identifiers are prefixed “\fileent{otfl@}” to
%    avoid name clashes.
%
%    \begin{macrocode}

do
    local new_attribute = luatexbase.new_attribute
    local the_attributes = luatexbase.attributes

    attributes = attributes or { }

    attributes.private = function (name)
        local attr   = "otfl@" .. name
        local number = the_attributes[attr]
        if not number then
            number = new_attribute(attr)
        end
        return number
    end
end

%    \end{macrocode}
%
%
%    \subsection{Callbacks}
%
%    After the fontloader is ready we can restore the callback trap from
%    \identifier{luatexbase}.
%
%    \begin{macrocode}

callback.register = trapped_register

%    \end{macrocode}
%
%    We do our own callback handling with the means provided by luatexbase.
%
%    Note: \luafunction{pre_linebreak_filter} and \luafunction{hpack_filter}
%    are coupled in \CONTEXT in the concept of \emphasis{node processor}.
%
%    \begin{macrocode}

add_to_callback("pre_linebreak_filter",
                generic_context.callback_pre_linebreak_filter,
                "luaotfload.node_processor",
                1)
add_to_callback("hpack_filter",
                generic_context.callback_hpack_filter,
                "luaotfload.node_processor",
                1)
add_to_callback("find_vf_file",
                find_vf_file, "luaotfload.find_vf_file")

loadmodule"font-otc.lua"   -- TODO check what we can drop from otfl-features
loadmodule"lib-dir.lua"    -- required by font-nms
loadmodule"luat-ovr.lua"

%    \end{macrocode}
%
%
%    \CONTEXT does not support ofm, these lines were added in order to make it
%    work. However they do not seem necessary so they are commented for now.
%
%    \begin{macrocode}
-- if fonts and fonts.readers.tfm then
--  fonts.readers.ofm  = fonts.readers.tfm
--  fonts.handlers.ofm = fonts.handlers.tfm --- empty anyways
--  fonts.formats.ofm  = fonts.formats.tfm  --- “type1”
--  --- fonts.readers.sequence[#fonts.readers.sequence+1] = "ofm"
--end
%    \end{macrocode}
%
%
%    Now we load the modules written for \identifier{luaotfload}.
%
%    \begin{macrocode}
loadmodule"font-pfb.lua"    --- new in 2.0, added 2011
loadmodule"font-nms.lua"
loadmodule"font-clr.lua"
loadmodule"font-ltx.lua"    --- new in 2.0, added 2011

%    \end{macrocode}
%
%
%    This hack makes fonts called with file method found by fonts.names.resove
%    instead of just trying to find them with kpse. It is necessary in case
%    of fonts that are not accessible by kpse but present in the database, a 
%    quite common case under Linux.
%
%    \begin{macrocode}

fonts.definers.resolvers.file = function (specification) 
  specification.name = fonts.names.resolve('', '', specification) 
end

%    \end{macrocode}
%
%
%    We create a callback for patching fonts on the fly, to be used by other
%    packages.
%    It initially contains the empty function that we are going to override
%    below.
%
%    \begin{macrocode}

create_callback("luaotfload.patch_font", "simple", dummy_function)

%    \end{macrocode}
%
%
%    This is a wrapper for the imported font loader.
%    As of 2013, everything it does appear to be redundand, so we won’t use
%    it unless somebody points out a cogent reason.
%    Nevertheless, it has been adapted to work with the current structure of
%    font data objects and will stay here for reference / until breakage is
%    reported.
%
%    \emphasis{TODO}
%    This one also enables patching fonts.
%    The current fontloader apparently comes with a dedicated mechanism for
%    that already: enhancers.
%    How those work remains to be figured out.
%
%    \begin{macrocode}
local define_font_wrapper = function (...)
    --- we use “tfmdata” (not “fontdata”) for consistency with the
    --- font loader
    local tfmdata = fonts.definers.read(...)
    if type(tfmdata) == "table" and tfmdata.shared then
        local metadata = tfmdata.shared.rawdata.metadata
        local mathdata = metadata.math --- do all fonts have this field?
        if mathdata then
            local mathconstants = { } --- why new hash, not modify in place?
            local units_per_em  = metadata.units_per_em
            local size          = tfmdata.size
            for k,v in next, mathdata do
                --- afaics this is alread taken care of by
                --- definers.read
                if stringfind(k, "Percent") then
                    -- keep percent values as is
                    print(k,v)
                    mathconstants[k] = v
                else
                    mathconstants[k] = v / units_per_em * size
                end
            end
            --- for \overwithdelims
            --- done by definers.read as well
            mathconstants.FractionDelimiterSize             = 1.01 * size
            --- fontloader has 2.4 × size
            mathconstants.FractionDelimiterDisplayStyleSize = 2.39 * size
            tfmdata.MathConstants = mathconstants
        end
        call_callback("luaotfload.patch_font", tfmdata)
    end
    return tfmdata
end

%    \end{macrocode}
%
%
%    \subsection{\CONTEXT override}
%
%    We provide a simplified version of the original font definition
%    callback.
%
%    \begin{macrocode}

local read_font_file = fonts.definers.read
local patch_defined_font = function (...)
    local tfmdata = read_font_file(...)-- spec -> size -> id -> tmfdata
    if type(tfmdata) == "table" then
        call_callback("luaotfload.patch_font", tfmdata)
    end
    return tfmdata
end

caches.compilemethod = "both"

reset_callback("define_font")

%    \end{macrocode}
%
%    Finally we register the callbacks
%
%    \begin{macrocode}

if luaotfload.font_definer == "old"  then
  add_to_callback("define_font",
                  define_font_wrapper,
                  "luaotfload.define_font",
                  1)
elseif luaotfload.font_definer == "generic"  then
  add_to_callback("define_font",
                  generic_context.callback_define_font,
                  "luaotfload.define_font",
                  1)
elseif luaotfload.font_definer == "patch"  then
  add_to_callback("define_font",
                  patch_defined_font,
                  "luaotfload.define_font",
                  1)
end

--[[todo--
--- The manual promises coercion of the file: lookup if
--- the asked name is enclosed in brackets.
--- A couple things make me doubt that this is the case:
---
---     1) there doesn’t appear to be code for these cases
---     2) the brackets remain part of the file name
---     3) we still get calls to names.resolve which
---        ignores the “lookup” field of the spec it gets
---
--- For this reason here is some code that a) coerces
--- file: lookups in these cases and b) strips the brackets
--- from the file name. As we *still* get name: lookups even
--- though this code is active I’ll just leave it here
--- for reference, ineffective as it is.
do
    local getspecification, makespecification =
        fonts.definers.getspecification, fonts.definers.makespecification

    local analyze = function (specification, size)
        local lookup, name, sub, method, detail = getspecification(specification or "")
        local filename = stringmatch(name, "^%[(.*)%]$")
        if filename then
            lookup   = "file"    --> coerce file:
            name     = filename  --> remove brackets
        end
        return makespecification(specification, lookup, name, sub, method, detail, size)
    end
    fonts.definers.analyze = analyze
end
--]]--

loadmodule"features.lua"

-- vim:tw=71:sw=4:ts=4:expandtab

%    \end{macrocode}
%
% \iffalse
%</lua>
% \fi
%
% \part{\fileent{luaotfload.sty}}
%
% \iffalse
%<*package>
% \fi
%
%    Classical Plain+\LATEX package initialization.
%
%    \begin{macrocode}
\csname ifluaotfloadloaded\endcsname
\let\ifluaotfloadloaded\endinput
\bgroup\expandafter\expandafter\expandafter\egroup
\expandafter\ifx\csname ProvidesPackage\endcsname\relax
  \input luatexbase.sty
\else
  \NeedsTeXFormat{LaTeX2e}
  \ProvidesPackage{luaotfload}%
    [2013/04/16 v2.2 OpenType layout system]
  \RequirePackage{luatexbase}
\fi
\RequireLuaModule{luaotfload}
\endinput
%    \end{macrocode}
% \iffalse
%</package>
% \fi
%
% \clearpage
% \section{The GNU GPL License v2}
%
% The GPL requires the complete license text to be distributed along
% with the code. I recommend the canonical source, instead:
% \url{http://www.gnu.org/licenses/old-licenses/gpl-2.0.html}.
% But if you insist on an included copy, here it is.
% You might want to zoom in.
%
% \newsavebox{\gpl}
% \begin{lrbox}{\gpl}
% \begin{minipage}{3\textwidth}
% \columnsep=3\columnsep
% \begin{multicols}{3}
% \begin{center}
% {\Large GNU GENERAL PUBLIC LICENSE\par}
% \bigskip
% {Version 2, June 1991}
% \end{center}
%
% \begin{center}
% {\parindent 0in
%
% Copyright \textcopyright\ 1989, 1991 Free Software Foundation, Inc.
%
% \bigskip
%
% 51 Franklin Street, Fifth Floor, Boston, MA  02110-1301, USA
%
% \bigskip
%
% Everyone is permitted to copy and distribute verbatim copies
% of this license document, but changing it is not allowed.
% }
% \end{center}
%
% \begin{center}
% {\bf\large Preamble}
% \end{center}
%
%
% The licenses for most software are designed to take away your freedom to
% share and change it.  By contrast, the GNU General Public License is
% intended to guarantee your freedom to share and change free software---to
% make sure the software is free for all its users.  This General Public
% License applies to most of the Free Software Foundation's software and to
% any other program whose authors commit to using it.  (Some other Free
% Software Foundation software is covered by the GNU Library General Public
% License instead.)  You can apply it to your programs, too.
%
% When we speak of free software, we are referring to freedom, not price.
% Our General Public Licenses are designed to make sure that you have the
% freedom to distribute copies of free software (and charge for this service
% if you wish), that you receive source code or can get it if you want it,
% that you can change the software or use pieces of it in new free programs;
% and that you know you can do these things.
%
% To protect your rights, we need to make restrictions that forbid anyone to
% deny you these rights or to ask you to surrender the rights.  These
% restrictions translate to certain responsibilities for you if you
% distribute copies of the software, or if you modify it.
%
% For example, if you distribute copies of such a program, whether gratis or
% for a fee, you must give the recipients all the rights that you have.  You
% must make sure that they, too, receive or can get the source code.  And
% you must show them these terms so they know their rights.
%
% We protect your rights with two steps: (1) copyright the software, and (2)
% offer you this license which gives you legal permission to copy,
% distribute and/or modify the software.
%
% Also, for each author's protection and ours, we want to make certain that
% everyone understands that there is no warranty for this free software.  If
% the software is modified by someone else and passed on, we want its
% recipients to know that what they have is not the original, so that any
% problems introduced by others will not reflect on the original authors'
% reputations.
%
% Finally, any free program is threatened constantly by software patents.
% We wish to avoid the danger that redistributors of a free program will
% individually obtain patent licenses, in effect making the program
% proprietary.  To prevent this, we have made it clear that any patent must
% be licensed for everyone's free use or not licensed at all.
%
% The precise terms and conditions for copying, distribution and
% modification follow.
%
% \begin{center}
% {\Large \sc Terms and Conditions For Copying, Distribution and
%   Modification}
% \end{center}
%
% \begin{enumerate}
% \item
% This License applies to any program or other work which contains a notice
% placed by the copyright holder saying it may be distributed under the
% terms of this General Public License.  The ``Program'', below, refers to
% any such program or work, and a ``work based on the Program'' means either
% the Program or any derivative work under copyright law: that is to say, a
% work containing the Program or a portion of it, either verbatim or with
% modifications and/or translated into another language.  (Hereinafter,
% translation is included without limitation in the term ``modification''.)
% Each licensee is addressed as ``you''.
%
% Activities other than copying, distribution and modification are not
% covered by this License; they are outside its scope.  The act of
% running the Program is not restricted, and the output from the Program
% is covered only if its contents constitute a work based on the
% Program (independent of having been made by running the Program).
% Whether that is true depends on what the Program does.
%
% \item You may copy and distribute verbatim copies of the Program's source
%   code as you receive it, in any medium, provided that you conspicuously
%   and appropriately publish on each copy an appropriate copyright notice
%   and disclaimer of warranty; keep intact all the notices that refer to
%   this License and to the absence of any warranty; and give any other
%   recipients of the Program a copy of this License along with the Program.
%
% You may charge a fee for the physical act of transferring a copy, and you
% may at your option offer warranty protection in exchange for a fee.
%
% \item
% You may modify your copy or copies of the Program or any portion
% of it, thus forming a work based on the Program, and copy and
% distribute such modifications or work under the terms of Section 1
% above, provided that you also meet all of these conditions:
%
% \begin{enumerate}
%
% \item 
% You must cause the modified files to carry prominent notices stating that
% you changed the files and the date of any change.
%
% \item
% You must cause any work that you distribute or publish, that in
% whole or in part contains or is derived from the Program or any
% part thereof, to be licensed as a whole at no charge to all third
% parties under the terms of this License.
%
% \item
% If the modified program normally reads commands interactively
% when run, you must cause it, when started running for such
% interactive use in the most ordinary way, to print or display an
% announcement including an appropriate copyright notice and a
% notice that there is no warranty (or else, saying that you provide
% a warranty) and that users may redistribute the program under
% these conditions, and telling the user how to view a copy of this
% License.  (Exception: if the Program itself is interactive but
% does not normally print such an announcement, your work based on
% the Program is not required to print an announcement.)
%
% \end{enumerate}
%
%
% These requirements apply to the modified work as a whole.  If
% identifiable sections of that work are not derived from the Program,
% and can be reasonably considered independent and separate works in
% themselves, then this License, and its terms, do not apply to those
% sections when you distribute them as separate works.  But when you
% distribute the same sections as part of a whole which is a work based
% on the Program, the distribution of the whole must be on the terms of
% this License, whose permissions for other licensees extend to the
% entire whole, and thus to each and every part regardless of who wrote it.
%
% Thus, it is not the intent of this section to claim rights or contest
% your rights to work written entirely by you; rather, the intent is to
% exercise the right to control the distribution of derivative or
% collective works based on the Program.
%
% In addition, mere aggregation of another work not based on the Program
% with the Program (or with a work based on the Program) on a volume of
% a storage or distribution medium does not bring the other work under
% the scope of this License.
%
% \item
% You may copy and distribute the Program (or a work based on it,
% under Section 2) in object code or executable form under the terms of
% Sections 1 and 2 above provided that you also do one of the following:
%
% \begin{enumerate}
%
% \item
%
% Accompany it with the complete corresponding machine-readable
% source code, which must be distributed under the terms of Sections
% 1 and 2 above on a medium customarily used for software interchange; or,
%
% \item
%
% Accompany it with a written offer, valid for at least three
% years, to give any third party, for a charge no more than your
% cost of physically performing source distribution, a complete
% machine-readable copy of the corresponding source code, to be
% distributed under the terms of Sections 1 and 2 above on a medium
% customarily used for software interchange; or,
%
% \item
%
% Accompany it with the information you received as to the offer
% to distribute corresponding source code.  (This alternative is
% allowed only for noncommercial distribution and only if you
% received the program in object code or executable form with such
% an offer, in accord with Subsection b above.)
%
% \end{enumerate}
%
%
% The source code for a work means the preferred form of the work for
% making modifications to it.  For an executable work, complete source
% code means all the source code for all modules it contains, plus any
% associated interface definition files, plus the scripts used to
% control compilation and installation of the executable.  However, as a
% special exception, the source code distributed need not include
% anything that is normally distributed (in either source or binary
% form) with the major components (compiler, kernel, and so on) of the
% operating system on which the executable runs, unless that component
% itself accompanies the executable.
%
% If distribution of executable or object code is made by offering
% access to copy from a designated place, then offering equivalent
% access to copy the source code from the same place counts as
% distribution of the source code, even though third parties are not
% compelled to copy the source along with the object code.
%
% \item
% You may not copy, modify, sublicense, or distribute the Program
% except as expressly provided under this License.  Any attempt
% otherwise to copy, modify, sublicense or distribute the Program is
% void, and will automatically terminate your rights under this License.
% However, parties who have received copies, or rights, from you under
% this License will not have their licenses terminated so long as such
% parties remain in full compliance.
%
% \item
% You are not required to accept this License, since you have not
% signed it.  However, nothing else grants you permission to modify or
% distribute the Program or its derivative works.  These actions are
% prohibited by law if you do not accept this License.  Therefore, by
% modifying or distributing the Program (or any work based on the
% Program), you indicate your acceptance of this License to do so, and
% all its terms and conditions for copying, distributing or modifying
% the Program or works based on it.
%
% \item
% Each time you redistribute the Program (or any work based on the
% Program), the recipient automatically receives a license from the
% original licensor to copy, distribute or modify the Program subject to
% these terms and conditions.  You may not impose any further
% restrictions on the recipients' exercise of the rights granted herein.
% You are not responsible for enforcing compliance by third parties to
% this License.
%
% \item
% If, as a consequence of a court judgment or allegation of patent
% infringement or for any other reason (not limited to patent issues),
% conditions are imposed on you (whether by court order, agreement or
% otherwise) that contradict the conditions of this License, they do not
% excuse you from the conditions of this License.  If you cannot
% distribute so as to satisfy simultaneously your obligations under this
% License and any other pertinent obligations, then as a consequence you
% may not distribute the Program at all.  For example, if a patent
% license would not permit royalty-free redistribution of the Program by
% all those who receive copies directly or indirectly through you, then
% the only way you could satisfy both it and this License would be to
% refrain entirely from distribution of the Program.
%
% If any portion of this section is held invalid or unenforceable under
% any particular circumstance, the balance of the section is intended to
% apply and the section as a whole is intended to apply in other
% circumstances.
%
% It is not the purpose of this section to induce you to infringe any
% patents or other property right claims or to contest validity of any
% such claims; this section has the sole purpose of protecting the
% integrity of the free software distribution system, which is
% implemented by public license practices.  Many people have made
% generous contributions to the wide range of software distributed
% through that system in reliance on consistent application of that
% system; it is up to the author/donor to decide if he or she is willing
% to distribute software through any other system and a licensee cannot
% impose that choice.
%
% This section is intended to make thoroughly clear what is believed to
% be a consequence of the rest of this License.
%
% \item
% If the distribution and/or use of the Program is restricted in
% certain countries either by patents or by copyrighted interfaces, the
% original copyright holder who places the Program under this License
% may add an explicit geographical distribution limitation excluding
% those countries, so that distribution is permitted only in or among
% countries not thus excluded.  In such case, this License incorporates
% the limitation as if written in the body of this License.
%
% \item
% The Free Software Foundation may publish revised and/or new versions
% of the General Public License from time to time.  Such new versions will
% be similar in spirit to the present version, but may differ in detail to
% address new problems or concerns.
%
% Each version is given a distinguishing version number.  If the Program
% specifies a version number of this License which applies to it and ``any
% later version'', you have the option of following the terms and conditions
% either of that version or of any later version published by the Free
% Software Foundation.  If the Program does not specify a version number of
% this License, you may choose any version ever published by the Free Software
% Foundation.
%
% \item
% If you wish to incorporate parts of the Program into other free
% programs whose distribution conditions are different, write to the author
% to ask for permission.  For software which is copyrighted by the Free
% Software Foundation, write to the Free Software Foundation; we sometimes
% make exceptions for this.  Our decision will be guided by the two goals
% of preserving the free status of all derivatives of our free software and
% of promoting the sharing and reuse of software generally.
%
% \begin{center}
% {\Large\sc
% No Warranty
% }
% \end{center}
%
% \item
% {\sc Because the program is licensed free of charge, there is no warranty
% for the program, to the extent permitted by applicable law.  Except when
% otherwise stated in writing the copyright holders and/or other parties
% provide the program ``as is'' without warranty of any kind, either expressed
% or implied, including, but not limited to, the implied warranties of
% merchantability and fitness for a particular purpose.  The entire risk as
% to the quality and performance of the program is with you.  Should the
% program prove defective, you assume the cost of all necessary servicing,
% repair or correction.}
%
% \item
% {\sc In no event unless required by applicable law or agreed to in writing
% will any copyright holder, or any other party who may modify and/or
% redistribute the program as permitted above, be liable to you for damages,
% including any general, special, incidental or consequential damages arising
% out of the use or inability to use the program (including but not limited
% to loss of data or data being rendered inaccurate or losses sustained by
% you or third parties or a failure of the program to operate with any other
% programs), even if such holder or other party has been advised of the
% possibility of such damages.}
%
% \end{enumerate}
%
%
% \begin{center}
% {\Large\sc End of Terms and Conditions}
% \end{center}
%
%
% \pagebreak[2]
%
% \section*{Appendix: How to Apply These Terms to Your New Programs}
%
% If you develop a new program, and you want it to be of the greatest
% possible use to the public, the best way to achieve this is to make it
% free software which everyone can redistribute and change under these
% terms.
%
%   To do so, attach the following notices to the program.  It is safest to
%   attach them to the start of each source file to most effectively convey
%   the exclusion of warranty; and each file should have at least the
%   ``copyright'' line and a pointer to where the full notice is found.
%
% \begin{quote}
% one line to give the program's name and a brief idea of what it does. \\
% Copyright (C) yyyy  name of author \\
%
% This program is free software; you can redistribute it and/or modify
% it under the terms of the GNU General Public License as published by
% the Free Software Foundation; either version 2 of the License, or
% (at your option) any later version.
%
% This program is distributed in the hope that it will be useful,
% but WITHOUT ANY WARRANTY; without even the implied warranty of
% MERCHANTABILITY or FITNESS FOR A PARTICULAR PURPOSE.  See the
% GNU General Public License for more details.
%
% You should have received a copy of the GNU General Public License
% along with this program; if not, write to the Free Software
% Foundation, Inc., 51 Franklin Street, Fifth Floor, Boston, MA  02110-1301, USA.
% \end{quote}
%
% Also add information on how to contact you by electronic and paper mail.
%
% If the program is interactive, make it output a short notice like this
% when it starts in an interactive mode:
%
% \begin{quote}
% Gnomovision version 69, Copyright (C) yyyy  name of author \\
% Gnomovision comes with ABSOLUTELY NO WARRANTY; for details type `show w'. \\
% This is free software, and you are welcome to redistribute it
% under certain conditions; type `show c' for details.
% \end{quote}
%
%
% The hypothetical commands {\tt show w} and {\tt show c} should show the
% appropriate parts of the General Public License.  Of course, the commands
% you use may be called something other than {\tt show w} and {\tt show c};
% they could even be mouse-clicks or menu items---whatever suits your
% program.
%
% You should also get your employer (if you work as a programmer) or your
% school, if any, to sign a ``copyright disclaimer'' for the program, if
% necessary.  Here is a sample; alter the names:
%
% \begin{quote}
% Yoyodyne, Inc., hereby disclaims all copyright interest in the program \\
% `Gnomovision' (which makes passes at compilers) written by James Hacker. \\
%
% signature of Ty Coon, 1 April 1989 \\
% Ty Coon, President of Vice
% \end{quote}
%
%
% This General Public License does not permit incorporating your program
% into proprietary programs.  If your program is a subroutine library, you
% may consider it more useful to permit linking proprietary applications
% with the library.  If this is what you want to do, use the GNU Library
% General Public License instead of this License.
%
% \end{multicols}
% \end{minipage}
% \end{lrbox}
%
% \begin{center}
% \scalebox{0.33}{\usebox{\gpl}}
% \end{center}
%
% \Finale
\endinput
